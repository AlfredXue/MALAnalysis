% v2-acmsmall-sample.tex, dated March 6 2012
% This is a sample file for ACM small trim journals
%
% Compilation using 'acmsmall.cls' - version 1.3 (March 2012), Aptara Inc.
% (c) 2010 Association for Computing Machinery (ACM)
%
% Questions/Suggestions/Feedback should be addressed to => "acmtexsupport@aptaracorp.com".
% Users can also go through the FAQs available on the journal's submission webpage.
%
% Steps to compile: latex, bibtex, latex latex
%
% For tracking purposes => this is v1.3 - March 2012

\documentclass[prodmode,acmtecs]{acmsmall} % Aptara syntax

% Package to generate and customize Algorithm as per ACM style
\usepackage[ruled]{algorithm2e}
\renewcommand{\algorithmcfname}{ALGORITHM}
\SetAlFnt{\small}
\SetAlCapFnt{\small}
\SetAlCapNameFnt{\small}
\SetAlCapHSkip{0pt}
\IncMargin{-\parindent}

% Metadata Information
\acmVolume{1}
\acmNumber{1}
\acmArticle{1}
\acmYear{2016}
\acmMonth{10}

% Document starts
\begin{document}

% Page heads
\markboth{Shi and Xue}{Non-Human Social Graphs -- a Case Study of MyAnimeList}

% Title portion
\title {Non-Human Social Graphs -- a Case Study of MyAnimeList}
\author{Wendy Shi
\affil{Stanford University}
Alfred Xue
\affil{Stanford University}}
% NOTE! Affiliations placed here should be for the institution where the
%       BULK of the research was done. If the author has gone to a new
%       institution, before publication, the (above) affiliation should NOT be changed.
%       The authors 'current' address may be given in the "Author's addresses:" block (below).
%       So for example, Mr. Abdelzaher, the bulk of the research was done at UIUC, and he is
%       currently affiliated with NASA.

\begin{abstract}
In CS224 lecture, we discussed the concept of a small world - the idea that the shortest path between any two individuals
is relatively small. This has been attributed to the principle that the likelihood that two individuals know each other
correlates significantly with the geographical distance between them. Effectively, this allows each step in the shortest
path to travel a percentage of the remaining distance, which causes the shortest path to grow logarithmically with 
geographical distance, rather than linearly. 

We are curious if these principles extend to web networks, which are generally considered to be networks that don't 
use humans as nodes, but can use their interactions to define edges. In particular, we want to study the degrees of separation 
between differing Anime, using ``myanimelist.net'' as our dataset. MyAnimeList (MAL) provides a unique opportunity to take a 
single set of nodes (anime), and multiple sets of edges that we can analyze. For example, edges can be recommendations 
from one anime to another, or a shared cast. 

\end{abstract}

\category{C.2.2}{Computer-Communication Networks}{Network Protocols}

\terms{Design, Algorithms, Performance}

\keywords{Small World, Network Analysis}

\acmformat{Wendy Shi, Alfred Xue, 2016. Diameter of MyAnimeList}
% At a minimum you need to supply the author names, year and a title.
% IMPORTANT:
% Full first names whenever they are known, surname last, followed by a period.
% In the case of two authors, 'and' is placed between them.
% In the case of three or more authors, the serial comma is used, that is, all author names
% except the last one but including the penultimate author's name are followed by a comma,
% and then 'and' is placed before the final author's name.
% If only first and middle initials are known, then each initial
% is followed by a period and they are separated by a space.
% The remaining information (journal title, volume, article number, date, etc.) is 'auto-generated'.

\begin{bottomstuff}
%This work is supported by the National Science Foundation, under
%grant CNS-0435060, grant CCR-0325197 and grant EN-CS-0329609.

%Author's addresses: G. Zhou, Computer Science Department,
%College of William and Mary; Y. Wu  {and} J. A. Stankovic,
%Computer Science Department, University of Virginia; T. Yan,
%Eaton Innovation Center; T. He, Computer Science Department,
%University of Minnesota; C. Huang, Google; T. F. Abdelzaher,
%(Current address) NASA Ames Research Center, Moffett Field, California 94035.
\end{bottomstuff}

\maketitle


\section{Introduction}
The question 
\begin{quote}
``If you chose any two people in the world at random, how many acquaintances are needed to create a chain between them?''
[Kochen, 1989; Garfield 1979] 
\end{quote} 
describes the small world problem in a succinct manner. Stanley Milgram's famous postcard experiment [citation???] was the 
first experiment that attempted to measure the degrees of separation between people in the world. To do so, he selected
a broker in Boston to be a target, and had a group of individuals send letters to acquaintances, seeing if those letters
could eventually be routed to the Boston broker. The famous "six degrees of separation" phrase is derived from the results 
of Milgram's experiment.

Although there are heavy criticisms around the veracity of Milgram's experiments, his work clearly demonstrates two ideas.
The first is that the degrees of separation in a human social graph are far smaller than one would intuitively think,
and the other is that the graph is \textit{navigable}, that is, there exists some algorithm that can traverse from any
node $s$ to destination node $t$ only knowing the edges of the current node in $O(\log(N)^\beta)$ time.

The intuition for both of these properties derives from the idea that the likelihood of two people knowing each other is strongly
correlated with their geographical distance. That is, in general, as long as the next hop is geographically closer, then our
new node has a shorter path to the destination node than our current one. However, this isn't sufficient to explain the
results of Milgram's experiments. For example, if every individual knows every other individual within a one mile radius, but
knows no one else, then the shortest path between two nodes eight hundred miles apart requires eight hundred hops -- order of 
magnitudes greater than what was observed in Milgram's study. Thus the second requirement is that with high probability, our current
node will be connected to another node whose distance to the destination is no more than $\alpha << 1$ times the distance from
our current node to the destination node. This allows us to travel to the destination much faster -- the number of hops now 
scales logarithmically rather than linearly with the distance to the destination. 

Although "six degrees of separation" is often cited, Milgram's experiment has a few key flaws that should prevent its results
from extending to the claim that "all humans are seperated by six degrees." These key flaws are highlighted by Judith Kleinfeld.
The first is that both the start and end nodes were recruited from socially active people -- 
basically, in expectation the degrees of separation between individuals is actually higher, because starting nodes and ending 
node should be less reachable than the ones used in the experiment. Other criticisms he highlights include that humans have
a psychological bias to believe in a short degree of separation. In general, Kleinfeld criticizes the legitimacy of the study, 
but does not provide any methods to provide a better study, and his criticisms probably aren't strong enough to reject the
concept of a small world, although they are probably strong enough to suggest that six degrees of separation is insufficient. 

Nevertheless, "six degrees of separation" is probably here to stay, and is supported by more recent analysis on the Facebook 
social network by Backstrom et. al. 

The Facebook network is clearly a social graph, and maintains the properties one would expect from a social graph.
Properties such as high clustering, low degrees of separation, and relatively low connections (i.e. number of edges << number of nodes$^2$) 
are all observed in this network. Other networks where these properties are observed includes a network of actors, where
edges are defined by having collaborated on multiple movies together, or a network of researchers, where an edge is
defined to be when two researches have collaborated on a published paper together. The work by Backstorm uses HyperANF, which
is similar to what snap.py uses to compute node distances, but uses hyperloglog counters to estimate these distances with
high accuracy. HyperANF increases the amount of nodes that can be computed on exponentially with little loss of overall accuracy 
(individual pairs are less accurate), and allows computation over 700 billion edges.


This study introduces the concept of a spid, or a \textit{index of dispersion} $\frac{\sigma^2}{\mu}$ of the distance distribution.
This paper describes the importance of spid as follows, 
\begin{quote}
    "In particular, networks with a spid larger than
    one should be considered “web-like”, whereas networks with a
    spid smaller than one should be considered 'properly social'"
\end{quote}
with the logic being that social networks tend to favor short connections (and thus low dispersion), whereas in web/information networks
long connections are not uncommon. The results of Backstorm's work clearly indicates that Facebook has a spid well below one. Because this
paper used HyperANF, it cannot compute the diameter on the graph -- it can only get a lower bound. This is one drawback of using 
HyperANF, and something to consider when working with large databases. One criticism of this work is that it does not attempt to draw
any meaningful conclusions. It concludes that Facebook is a "small-world graph" with an average distance of less than 4, but does not
attempt to understand any interesting phenomenon observed. We would also liked to have seen analysis on the diameter of Facebook, but 
that may be infeasible with the computing power at the time.


At this point it becomes clear that there is an interesting distinction between social graphs and web graphs, so we would like to explore
the distinction between them. Luckily there has already been work done on this on the twitter graph by Myers et. al. Myers et. al. wanted to 
determine whether Twitter acted more like a social graph or an information network. The basis of this question is that follows are done 
through interests rather than relationships, which makes it more like an information network, but initial follows are often based on 
social interactions. They define the nodes to be twitter accounts and edges to be mutual follows between the accounts.


Myers et. al. also used HyperANF on their dataset of 175 million users and approximately twenty billion edges for mutual follows. (Twitter 
has more edges per node than Facebook since follows are more common than friendships). In this case, they analyzed multiple graph characteristics
of the Twitter graph and compared them to the known characteristics of both information and social graphs. For degree distributions, it is clear
that Twitter does not exhibit social graph like characteristics because it has too many individuals who follow and are followed by thousands 
of people when people are typically unable to maintain more than 150 social relationships at any given time. For both shortest path length and spid,
the authors concluded that Twitter exhibited social graph like behaviors, with the shortest path being around 4 (slightly longer than Facebook), and
the spid being around 0.11 (slightly wider than Facebook). They also noted that the clustering coefficient of the Twitter graph was relatively high,
and thus indicative of a social graph. However, they also noted that the two-hop neighborhoods of Twitter were much much larger than what 
would be expected in a social graph. 

The authors than unhelpfully concluded that Twitter exhibits some characteristics of a social graph and some characteristics of a information graph,
but make no attempt to unify the differing characteristics. 


It is in this space that we think interesting work can be done. 
\section{Project Proposal}
From the above papers we derive two interesting questions: Are social graphs only apparent when nodes are human? Can we use graph characteristics to 
determine how "social" different edge types are? 


The first comes from the fact that all the papers we have read so far on social graphs always use human nodes -- people in a friend 
graph like Facebook, or actors or researchers in a collaboration graph, etc. We are curious if a graph can exhibit social graph features when
the nodes are non-human but the edges are derived from humans. One example of such a graph could be research papers that share an edge if they
share a mutual author. 


The Twitter paper discusses its decision to use mutual follows to generate edges but does some analysis on inbound/outbound only edges. Through this,
it becomes clear that (and is rather intuitive) that the sociality of a graph is highly dependent on the type of edges used. 


To research both these areas, we propose the following project:

Analyze the graph characteristics of MyAnimeList using different Anime as nodes, and a differing set of edges including recommendations 
and actorship.

The graph characteristics we will analyze are the ones that were used to analyze the twitter dataset, specifically degree distributions, shortest path length,
spid, clustering coefficient, and two-hop neighborhoods. We can also add the diameter of the graph, which was largely ignored by Twitter due to the expensiveness
of its calculation. If necessary, we may also analyze subgraphs of MyAnimeList dividing by genre.

\subsection{Background on MyAnimeList}
MyAnimeList is a website that has aggregate data about tens of thousands of anime. For each anime, it provides metadata information such as genres, ratings, and
episode length, as well as voice actors and user recommendations. 

There are currently 34240 anime in the database, and around fifteen recommendations per anime, although this varies greatly depending on the popularity of 
the anime. 

There is also user information, where users can input the animes that they have watched and their ratings of them, which can also be used as an edge set, although
we have not tried scrapping this yet. 

To scrap the Anime data from MyAnimeList, we are modifying an existing python library https://pypi.python.org/pypi/mal-scraper/0.1.0 that no longer works 
due to MyAnimeList changing their tags.
\subsection{Algorithms for Analysis} 
Luckily, our dataset is not a particularly huge one, so it is unlikely that we will need complex algorithms such as HyperANF to determine our graph characteristics.
Degree distribution calculation does not require a particularly complex algorithm, and our graph is small enough that storing a value for every node is not expensive. 
Both connected component and shortest length pass can be calculated using snap functions. A trivial implementation of two-hop neighborhoods is functionally an $O(VE)$ 
operation (get set of nodes at one-hop, get set of nodes at two-hop) per node, or a $O(V^2E)$ operation in total. At 30,000 nodes, this is also not expensive. The Floyd-Warshall
algorithm can compute the diameter of the graph in $O(V^3)$, which is sufficient for our small graph. 
\subsection{Evaluation} 
Like in the Twitter paper, we will compare the graph characteristics of MyAnimeList to a standard social graph to determine which characteristics of a social graph MyAnimeList exhibits.
However, we will go one step further and try to explain why only some of the features of a social graph are represented (if that is indeed the case), and attempt to weigh which 
features are more reflective of a social graph than others.
We will then compare the characteristics of MyAnimeList across edge sets, and attempt to explain why different edge sets produce different graph characteristics. 
We can then verify our theories of why edge sets produce specific graph characteristics by perturbing the edge set and then comparing graph charactersitics. For example, if our theory is that 
MyAnimeList doesn't exhibit a social graph because the average Anime has too many recommendations, we can perturb the graph so that we only include edges between Anime when there have 
been at least two recommendations between them. 
We define success to be when we have found a theory of why MyAnimeList exhibits a set of graph characteristics that we can demonstrate the validity of the theory through edge perturbation.
\subsection{Submission}
At the end of the quarter, we expect to be able to submit our MyAnimeList scrapping code and a paper that both analyzes the graph characteristics of MyAnimeList and provides a theory as to
what real-life properties of edges generate what graph characteristics. 
\section{Typical references in new ACM Reference Format}
A paginated journal article \cite{Abril07}, an enumerated
journal article \cite{Cohen07}, a reference to an entire issue \cite{JCohen96},
a monograph (whole book) \cite{Kosiur01}, a monograph/whole book in a series (see 2a in spec. document)
\cite{Harel79}, a divisible-book such as an anthology or compilation \cite{Editor00}
followed by the same example, however we only output the series if the volume number is given
\cite{Editor00a} (so Editor00a's series should NOT be present since it has no vol. no.),
a chapter in a divisible book \cite{Spector90}, a chapter in a divisible book
in a series \cite{Douglass98}, a multi-volume work as book \cite{Knuth97},
an article in a proceedings (of a conference, symposium, workshop for example)
(paginated proceedings article) \cite{Andler79}, a proceedings article
with all possible elements \cite{Smith10}, an example of an enumerated
proceedings article \cite{VanGundy07},
an informally published work \cite{Harel78}, a doctoral dissertation \cite{Clarkson85},
a master's thesis: \cite{anisi03}, an online document / world wide web resource \cite{Thornburg01}, \cite{Ablamowicz07},
\cite{Poker06}, a video game (Case 1) \cite{Obama08} and (Case 2) \cite{Novak03}
and \cite{Lee05} and (Case 3) a patent \cite{JoeScientist001},
work accepted for publication \cite{rous08}, 'YYYYb'-test for prolific author
\cite{SaeediMEJ10} and \cite{SaeediJETC10}. Other cites might contain
'duplicate' DOI and URLs (some SIAM articles) \cite{Kirschmer:2010:AEI:1958016.1958018}.
Boris / Barbara Beeton: multi-volume works as books
\cite{MR781536} and \cite{MR781537}.

% Appendix
\appendix
\section*{APPENDIX}
\setcounter{section}{1}
In this appendix, we measure
the channel switching time of Micaz [CROSSBOW] sensor devices.
In our experiments, one mote alternatingly switches between Channels
11 and 12. Every time after the node switches to a channel, it sends
out a packet immediately and then changes to a new channel as soon
as the transmission is finished. We measure the
number of packets the test mote can send in 10 seconds, denoted as
$N_{1}$. In contrast, we also measure the same value of the test
mote without switching channels, denoted as $N_{2}$. We calculate
the channel-switching time $s$ as
\begin{eqnarray}%
s=\frac{10}{N_{1}}-\frac{10}{N_{2}}. \nonumber
\end{eqnarray}%
By repeating the experiments 100 times, we get the average
channel-switching time of Micaz motes: 24.3$\mu$s.

\appendixhead{ZHOU}

% Acknowledgments
\begin{acks}
The authors would like to thank Dr. Maura Turolla of Telecom
Italia for providing specifications about the application scenario.
\end{acks}

% Bibliography
\bibliographystyle{ACM-Reference-Format-Journals}
\bibliography{acmsmall-sample-bibfile}
                             % Sample .bib file with references that match those in
                             % the 'Specifications Document (V1.5)' as well containing
                             % 'legacy' bibs and bibs with 'alternate codings'.
                             % Gerry Murray - March 2012

% History dates
\received{February 2007}{March 2009}{June 2009}

% Electronic Appendix
\elecappendix

\medskip

\section{This is an example of Appendix section head}

Channel-switching time is measured as the time length it takes for
motes to successfully switch from one channel to another. This
parameter impacts the maximum network throughput, because motes
cannot receive or send any packet during this period of time, and it
also affects the efficiency of toggle snooping in MMSN, where motes
need to sense through channels rapidly.

By repeating experiments 100 times, we get the average
channel-switching time of Micaz motes: 24.3 $\mu$s. We then conduct
the same experiments with different Micaz motes, as well as
experiments with the transmitter switching from Channel 11 to other
channels. In both scenarios, the channel-switching time does not have
obvious changes. (In our experiments, all values are in the range of
23.6 $\mu$s to 24.9 $\mu$s.)

\section{Appendix section head}

The primary consumer of energy in WSNs is idle listening. The key to
reduce idle listening is executing low duty-cycle on nodes. Two
primary approaches are considered in controlling duty-cycles in the
MAC layer.

\end{document}
% End of v2-acmsmall-sample.tex (March 2012) - Gerry Murray, ACM


