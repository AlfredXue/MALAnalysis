% v2-acmsmall-sample.tex, dated March 6 2012
% This is a sample file for ACM small trim journals
%
% Compilation using 'acmsmall.cls' - version 1.3 (March 2012), Aptara Inc.
% (c) 2010 Association for Computing Machinery (ACM)
%
% Questions/Suggestions/Feedback should be addressed to => "acmtexsupport@aptaracorp.com".
% Users can also go through the FAQs available on the journal's submission webpage.
%
% Steps to compile: latex, bibtex, latex latex
%
% For tracking purposes => this is v1.3 - March 2012

\documentclass[prodmode,acmtecs]{acmsmall} % Aptara syntax

% Package to generate and customize Algorithm as per ACM style
\usepackage[ruled]{algorithm2e}
\renewcommand{\algorithmcfname}{ALGORITHM}
\SetAlFnt{\small}
\SetAlCapFnt{\small}
\SetAlCapNameFnt{\small}
\SetAlCapHSkip{0pt}
\IncMargin{-\parindent}

% Metadata Information
\acmVolume{1}
\acmNumber{1}
\acmArticle{1}
\acmYear{2016}
\acmMonth{10}

% Document starts
\begin{document}

% Page heads
\markboth{Shi and Xue}{Non-Human Social Graphs} 

% Title portion
\title {Non-Human Social Graphs}
\author{Wendy Shi
\affil{Stanford University}
Alfred Xue
\affil{Stanford University}}

\begin{abstract}
In CS224 lecture, we discussed the concept of a small world - the idea that the shortest path between any two individuals
is relatively small. This has been attributed to the principle that the likelihood that two individuals know each other
correlates significantly with the geographical distance between them. Effectively, this allows each step in the shortest
path to travel a percentage of the remaining distance, which causes the shortest path to grow logarithmically with 
geographical distance, rather than linearly. Other properties, such as having small two-hop neighborhoods and a specific
degree distribution have all been attributed to Social Network. 

We are curious to see which of these characteristics can be found in non-social graphs, and using such information evaluate
which of these characteristics are actually specific to social graphs, and which might be artifacts of some property that 
isn't specific to social graphs. To do so, we compute the coefficents for these characteristics on many social and information 
networks, including MyAnimeList, actors, and Wikipedia, and compare these values to those found on Social Networks such as 
Facebook. We found that (insert conclusion here). 
\end{abstract}

\category{C}{Graph Characteristics}{}

\terms{Design, Algorithms, Performance}

\keywords{Small World, Network Analysis}

\acmformat{Wendy Shi, Alfred Xue, 2016. Diameter of MyAnimeList}
\begin{bottomstuff}
\end{bottomstuff}

\maketitle


\section{Introduction}
Plenty of research has been done on the topological properties of social and information networks. From a non-technical perspective, 
it is very easy to define what a social network is. We think of it as the networks that describe some types of relationships between 
individuals in society. The most popular example is Facebook, which defines itself to be a social network. But there are also other 
social networks, such as phone contacts and collaborators. Information networks tend to be defined as "not social networks", but we
think of them as web links or co-authored research papers. Myers et al admits that there isn't a clear definition by researchers of
what a social network is, and their own research on the Twitter network demonstrates that the boundary between a social network and
a non-social network isn't super clear. Our objective is to evaluate the topological properties of a set of networks that have varrying degrees 
of what people would traditionally consider to be a "social network", and use those to attempt to discern which properties are more 
relevant to a technical definition of a social network.

The networks that we studied (from most social to most informational) are Facebook, Actors, Twitter, MyAnimeList with edges based on recommendations,
MyAnimeList with edges based on Genre, and Wikipedia page links. 

\section{Literature Review}
The question 
\begin{quote}
``If you chose any two people in the world at random, how many acquaintances are needed to create a chain between them?''
[Kochen, 1989; Garfield 1979] 
\end{quote} 
describes the small world problem in a succinct manner. Stanley Milgram's famous postcard experiment was the 
first experiment that attempted to measure the degrees of separation between people in the world. To do so, he selected
a broker in Boston to be a target, and had a group of individuals send letters to acquaintances, seeing if those letters
could eventually be routed to the Boston broker. The famous "six degrees of separation" phrase is derived from the results 
of Milgram's experiment.

Although there are heavy criticisms around the veracity of Milgram's experiments, his work clearly demonstrates two ideas.
The first is that the degrees of separation in a human social graph are far smaller than one would intuitively think,
and the other is that the graph is \textit{navigable}, that is, there exists some algorithm that can traverse from any
node $s$ to destination node $t$ only knowing the edges of the current node in $O(\log(N)^\beta)$ time.

The intuition for both of these properties derives from the idea that the likelihood of two people knowing each other is strongly
correlated with their geographical distance. That is, in general, as long as the next hop is geographically closer, then our
new node has a shorter path to the destination node than our current one. However, this isn't sufficient to explain the
results of Milgram's experiments. For example, if every individual knows every other individual within a one mile radius, but
knows no one else, then the shortest path between two nodes eight hundred miles apart requires eight hundred hops -- order of 
magnitudes greater than what was observed in Milgram's study. Thus the second requirement is that with high probability, our current
node will be connected to another node whose distance to the destination is no more than $\alpha << 1$ times the distance from
our current node to the destination node. This allows us to travel to the destination much faster -- the number of hops now 
scales logarithmically rather than linearly with the distance to the destination. 

Although "six degrees of separation" is often cited, Milgram's experiment has a few key flaws that should prevent its results
from extending to the claim that "all humans are seperated by six degrees." These key flaws are highlighted by Judith Kleinfeld.
The first is that both the start and end nodes were recruited from socially active people -- 
basically, in expectation the degrees of separation between individuals is actually higher, because starting nodes and ending 
node should be less reachable than the ones used in the experiment. Other criticisms he highlights include that humans have
a psychological bias to believe in a short degree of separation. In general, Kleinfeld criticizes the legitimacy of the study, 
but does not provide any methods to provide a better study, and his criticisms probably aren't strong enough to reject the
concept of a small world, although they are probably strong enough to suggest that six degrees of separation is insufficient [1].

Nevertheless, "six degrees of separation" is probably here to stay, and is supported by more recent analysis on the Facebook 
social network by Backstrom et. al. 

The Facebook network is clearly a social graph, and maintains the properties one would expect from a social graph.
Properties such as high clustering, low degrees of separation, and relatively low connections (i.e. number of edges << number of nodes$^2$) 
are all observed in this network. Other networks where these properties are observed includes a network of actors, where
edges are defined by having collaborated on multiple movies together, or a network of researchers, where an edge is
defined to be when two researches have collaborated on a published paper together. The work by Backstorm uses HyperANF, which
is similar to what snap.py uses to compute node distances, but uses hyperloglog counters to estimate these distances with
high accuracy. HyperANF increases the amount of nodes that can be computed on exponentially with little loss of overall accuracy 
(individual pairs are less accurate), and allows computation over 700 billion edges [2].


This study introduces the concept of a spid, or a \textit{index of dispersion} $\frac{\sigma^2}{\mu}$ of the distance distribution.
This paper describes the importance of spid as follows, 
\begin{quote}
    "In particular, networks with a spid larger than
    one should be considered “web-like”, whereas networks with a
    spid smaller than one should be considered 'properly social'"
\end{quote}
with the logic being that social networks tend to favor short connections (and thus low dispersion), whereas in web/information networks
long connections are not uncommon. The results of Backstorm's work clearly indicates that Facebook has a spid well below one. Because this
paper used HyperANF, it cannot compute the diameter on the graph -- it can only get a lower bound. This is one drawback of using 
HyperANF, and something to consider when working with large databases. One criticism of this work is that it does not attempt to draw
any meaningful conclusions. It concludes that Facebook is a "small-world graph" with an average distance of less than 4, but does not
attempt to understand any interesting phenomenon observed. We would also liked to have seen analysis on the diameter of Facebook, but 
that may be infeasible with the computing power at the time.


At this point it becomes clear that there is an interesting distinction between social graphs and web graphs, so we would like to explore
the distinction between them. Luckily there has already been work done on this on the twitter graph by Myers et. al. Myers et. al. wanted to 
determine whether Twitter acted more like a social graph or an information network. The basis of this question is that follows are done 
through interests rather than relationships, which makes it more like an information network, but initial follows are often based on 
social interactions. They define the nodes to be twitter accounts and edges to be mutual follows between the accounts.


Myers et. al. also used HyperANF on their dataset of 175 million users and approximately twenty billion edges for mutual follows. (Twitter 
has more edges per node than Facebook since follows are more common than friendships). In this case, they analyzed multiple graph characteristics
of the Twitter graph and compared them to the known characteristics of both information and social graphs. For degree distributions, it is clear
that Twitter does not exhibit social graph like characteristics because it has too many individuals who follow and are followed by thousands 
of people when people are typically unable to maintain more than 150 social relationships at any given time. For both shortest path length and spid,
the authors concluded that Twitter exhibited social graph like behaviors, with the shortest path being around 4 (slightly longer than Facebook), and
the spid being around 0.11 (slightly wider than Facebook). They also noted that the clustering coefficient of the Twitter graph was relatively high,
and thus indicative of a social graph. However, they also noted that the two-hop neighborhoods of Twitter were much much larger than what 
would be expected in a social graph. 

The authors than unhelpfully concluded that Twitter exhibits some characteristics of a social graph and some characteristics of a information graph,
but make no attempt to unify the differing characteristics [3].

% Bibliography
\section{Bibliography}
[1] Kleinfield, J. 2002. Could It Be A Big World?. Forthcoming Society 2002.

[2] L. Backstrom, P. Boldi, M. Rosa, J. Ugander, and S.Vigna. Four degrees of separation. WebSci 2012.

[3] S. A. Myers, A. Sharma, P. Gupta, and J. Lin. Information network or social network? The structure of the Twitter follow graph. WWW Companion, 2014.
\end{document}
% End of v2-acmsmall-sample.tex (March 2012) - Gerry Murray, ACM


